%============================================================================%
%
%	DOCUMENT DEFINITION
%
%============================================================================%

\documentclass[10pt,A4,english]{article}	


%----------------------------------------------------------------------------------------
%	ENCODING
%----------------------------------------------------------------------------------------

% we use utf8 since we want to build from any machine
\usepackage[utf8]{inputenc}		
\usepackage[USenglish]{isodate}
\usepackage{fancyhdr}
\usepackage[numbers]{natbib}

%badranx
\usepackage{fontawesome5}
\usepackage{enumitem}
\setlist{nolistsep}
%end badranx

%----------------------------------------------------------------------------------------
%	LOGIC
%----------------------------------------------------------------------------------------

% provides \isempty test
\usepackage{xstring, xifthen}
\usepackage{enumitem}
\usepackage[english]{babel}
\usepackage{blindtext}
\usepackage{pdfpages}
\usepackage{changepage}
%----------------------------------------------------------------------------------------
%	FONT BASICS
%----------------------------------------------------------------------------------------

% some tex-live fonts - choose your own

%\usepackage[defaultsans]{droidsans}
%\usepackage[default]{comfortaa}
%\usepackage{cmbright}
\usepackage[default]{raleway}
%\usepackage{fetamont}
%\usepackage[default]{gillius}
%\usepackage[light,math]{iwona}
%\usepackage[thin]{roboto} 

% set font default
\renewcommand*\familydefault{\sfdefault} 	
\usepackage[T1]{fontenc}

% more font size definitions
\usepackage{moresize}

%----------------------------------------------------------------------------------------
%	FONT AWESOME ICONS
%---------------------------------------------------------------------------------------- 

% use to vertically center content
% credits to: http://tex.stackexchange.com/questions/7219/how-to-vertically-center-two-images-next-to-each-other
\newcommand{\vcenteredinclude}[1]{\begingroup
\setbox0=\hbox{\includegraphics{#1}}%
\parbox{\wd0}{\box0}\endgroup}
\newcommand{\tab}[1]{\hspace{.2\textwidth}\rlap{#1}}
% use to vertically center content
% credits to: http://tex.stackexchange.com/questions/7219/how-to-vertically-center-two-images-next-to-each-other
\newcommand*{\vcenteredhbox}[1]{\begingroup
\setbox0=\hbox{#1}\parbox{\wd0}{\box0}\endgroup}

% icon shortcut
\newcommand{\icon}[3] { 							
	\makebox(#2, #2){\textcolor{maincol}{\csname fa#1\endcsname}}
}	


% icon with text shortcut
\newcommand{\icontext}[4]{ 						
	\vcenteredhbox{\icon{#1}{#2}{#3}}  \hspace{2pt}  \parbox{0.9\mpwidth}{\textcolor{#4}{#3}}
}

% icon with website url
\newcommand{\iconhref}[5]{ 						
    \vcenteredhbox{\icon{#1}{#2}{#5}}  \hspace{2pt} \href{#4}{\textcolor{#5}{#3}}
}

% icon with email link
\newcommand{\iconemail}[5]{ 						
    \vcenteredhbox{\icon{#1}{#2}{#5}}  \hspace{2pt} \href{mailto:#4}{\textcolor{#5}{#3}}
}

%----------------------------------------------------------------------------------------
%	PAGE LAYOUT  DEFINITIONS
%----------------------------------------------------------------------------------------

% page outer frames (debug-only)
% \usepackage{showframe}		

% we use paracol to display breakable two columns
\usepackage{paracol}
\usepackage{tikzpagenodes}
\usetikzlibrary{calc}
\usepackage{lmodern}
\usepackage{multicol}
\usepackage{lipsum}
\usepackage{atbegshi}
% define page styles using geometry
\usepackage[a4paper]{geometry}

% remove all possible margins
%originally all are 1cm
\geometry{top=0.7cm, bottom=0.5cm, left=0.5cm, right=1cm}

\usepackage{fancyhdr}
\pagestyle{empty}

% space between header and content
% \setlength{\headheight}{0pt}

% indentation is zero
\setlength{\parindent}{0mm}

%----------------------------------------------------------------------------------------
%	TABLE /ARRAY DEFINITIONS
%---------------------------------------------------------------------------------------- 

% extended aligning of tabular cells
\usepackage{array}

% custom column right-align with fixed width
% use like p{size} but via x{size}
\newcolumntype{x}[1]{%
>{\raggedleft\hspace{0pt}}p{#1}}%


%----------------------------------------------------------------------------------------
%	GRAPHICS DEFINITIONS
%---------------------------------------------------------------------------------------- 

%for header image
\usepackage{graphicx}

% use this for floating figures
% \usepackage{wrapfig}
% \usepackage{float}
% \floatstyle{boxed} 
% \restylefloat{figure}

%for drawing graphics		
\usepackage{tikz}			
\usepackage{ragged2e}	
\usetikzlibrary{shapes, backgrounds,mindmap, trees}

%----------------------------------------------------------------------------------------
%	Color DEFINITIONS
%---------------------------------------------------------------------------------------- 
\usepackage{transparent}
\usepackage{color}

% primary color
\definecolor{maincol}{RGB}{ 64,64,64}

% accent color, secondary
% \definecolor{accentcol}{RGB}{ 250, 150, 10 }

% dark color
\definecolor{darkcol}{RGB}{ 70, 70, 70 }
%\definecolor{darkcol}{RGB}{146,166,219}
%\definecolor{darkcol}{RGB}{	0, 112, 224}
% 	(146,166,219)
% light color
\definecolor{lightcol}{RGB}{245,245,245}
%\definecolor{lightcol}{RGB}{255,0,0}

\definecolor{accentcol}{RGB}{59,77,97}
%\definecolor{accentcol}{RGB}{0,0,0}



% Package for links, must be the last package used
\usepackage[hidelinks]{hyperref}

% returns minipage width minus two times \fboxsep
% to keep padding included in width calculations
% can also be used for other boxes / environments
\newcommand{\mpwidth}{\linewidth-\fboxsep-\fboxsep}
	


%============================================================================%
%
%	CV COMMANDS
%
%============================================================================%

%----------------------------------------------------------------------------------------
%	 CV LIST
%----------------------------------------------------------------------------------------

% renders a standard latex list but abstracts away the environment definition (begin/end)
\newcommand{\cvlist}[1] {
	\begin{itemize}{#1}\end{itemize}
}

%----------------------------------------------------------------------------------------
%	 CV TEXT
%----------------------------------------------------------------------------------------

% base class to wrap any text based stuff here. Renders like a paragraph.
% Allows complex commands to be passed, too.
% param 1: *any
\newcommand{\cvtext}[1] {
	\begin{tabular*}{1\mpwidth}{p{0.98\mpwidth}}
		\parbox{1\mpwidth}{#1}
	\end{tabular*}
}
\newcommand{\cvtextsmall}[1] {
	\begin{tabular*}{0.8\mpwidth}{p{0.8\mpwidth}}
		\parbox{0.8\mpwidth}{#1}
	\end{tabular*}
}
%----------------------------------------------------------------------------------------
%	CV SECTION
%----------------------------------------------------------------------------------------

% Renders a a CV section headline with a nice underline in main color.
% param 1: section title
\newcommand{\cvsection}[1] {
	%\vspace{14pt}
	\cvtext{
		\textbf{\Large{\textcolor{darkcol}{#1}}}\\[-4pt]
		\textcolor{accentcol}{ \rule{0.2\textwidth}{1.5pt} } \\
	}
}

% Renders a a CV section headline with a nice underline in main color.
% param 1: section title
\newcommand{\cvsectionTight}[1] {
	%\vspace{14pt}
	\cvtext{
		\textbf{\Large{\textcolor{darkcol}{#1}}}\\[-4pt]
		\textcolor{accentcol}{ \rule{0.2\textwidth}{1.5pt} }% \\
	}
	\vspace{4pt}
}

\newcommand{\cvsectionsmall}[1] {
	\vspace{14pt}
	\cvtext{
		\textbf{\Large{\textcolor{darkcol}{#1}}}\\[-4pt]
		\textcolor{accentcol}{ \rule{0.2\textwidth}{1.5pt} } \\
	}
}

%badran begin
\newcommand{\cvsectionsmallsmall}[1] {
	\cvtext{
    	\vspace{18pt}
		\textbf{\large{\textcolor{darkcol}{#1}}} \\[-4pt]
		\textcolor{accentcol}{ \rule{0.2\textwidth}{0.5pt} } 
    	\vspace{6pt}
	}
	
}

%badran end

\newcommand{\cvheadline}[1] {
	\vspace{16pt}
	\cvtext{
		\textbf{\Huge{\textcolor{accentcol}{#1}}}\\[-4pt]
		 
	}
}

\newcommand{\cvsubheadline}[1] {
	\vspace{16pt}
	\cvtext{
		\textbf{\huge{\textcolor{darkcol}{#1}}}\\[-4pt]
		 
	}
}
%----------------------------------------------------------------------------------------
%	META SKILL
%----------------------------------------------------------------------------------------

% Renders a progress-bar to indicate a certain skill in percent.
% param 1: name of the skill / tech / etc.
% param 2: level (for example in years)
% param 3: percent, values range from 0 to 1
\newcommand{\cvskill}[3] {
	\begin{tabular*}{1\mpwidth}{p{0.72\mpwidth}  r}
 		\textcolor{black}{\textbf{#1}} & \textcolor{maincol}{#2}\\
	\end{tabular*}%
	
	\hspace{4pt}
	\begin{tikzpicture}[scale=1,rounded corners=2pt,very thin]
		\fill [lightcol] (0,0) rectangle (1\mpwidth, 0.15);
		\fill [accentcol] (0,0) rectangle (#3\mpwidth, 0.15);
  	\end{tikzpicture}%
}

\newcommand{\cvskillsection}[1]{
\cvtext{
    %\vspace{8pt}
    \textcolor{black}{\textbf{#1}} 
    \vspace{2pt}
}
}


\newcommand{\techno}[1]{
\textcolor{maincol}{\footnotesize Technology: #1}
}
%----------------------------------------------------------------------------------------
%	 CV EVENT
%----------------------------------------------------------------------------------------

% Renders a table and a paragraph (cvtext) wrapped in a parbox (to ensure minimum content
% is glued together when a pagebreak appears).
% Additional Information can be passed in text or list form (or other environments).
% the work you did
% param 1: time-frame i.e. Sep 14 - Jan 15 etc.
% param 2:	 event name (job position etc.)
% param 3: Customer, Employer, Industry
% param 4: Short description
% param 5: work done (optional)
% param 6: technologies include (optional)
% param 7: achievements (optional)
\newcommand{\cvevent}[7] {
	% we wrap this part in a parbox, so title and description are not separated on a pagebreak
	% if you need more control on page breaks, remove the parbox
	\parbox{\mpwidth}{
		\begin{tabular*}{1\mpwidth}{p{0.66\mpwidth}  r}
	 		\textcolor{black}{\textbf{#2}} & \colorbox{accentcol}{\makebox[0.32\mpwidth]{\textcolor{white}{\textbf{#1}}}} 
			\ifthenelse{\isempty{#3}}{}{\\
			\textcolor{maincol}{#3}% &
			} %\\
			\ifthenelse{\isempty{#4}}{}{\\
			\multicolumn{2}{l}{\techno{#4} }
			} %\\
		\end{tabular*}\\[3pt]
		\ifthenelse{\isempty{#5}}{\vspace{8pt}}{
			\cvtext{#5\vspace{14pt}}\\
		}
	}
	%\vspace{14pt}
}


%----------------------------------------------------------------------------------------
%	 CV META EVENT
%----------------------------------------------------------------------------------------

% Renders a CV event on the sidebar
% param 1: title
% param 2: subtitle (optional)
% param 3: customer, employer, etc,. (optional)
% param 4: info text (optional)
\newcommand{\cvmetaevent}[4] {
	\textcolor{maincol} { \cvtext{\textbf{\begin{flushleft}#1\end{flushleft}}}}

	\ifthenelse{\isempty{#2}}{}{
	\textcolor{black} {\cvtext{\textbf{#2}} }
	}

	\ifthenelse{\isempty{#3}}{}{
		\cvtext{{ \textcolor{maincol} {#3} }}\\
	}

	\cvtext{#4}\\[14pt]
}

%---------------------------------------------------------------------------------------
%	QR CODE
%----------------------------------------------------------------------------------------

% Renders a qrcode image (centered, relative to the parentwidth)
% param 1: percent width, from 0 to 1
\newcommand{\cvqrcode}[1] {
	\begin{center}
		\includegraphics[width={#1}\mpwidth]{qrcode}
	\end{center}
}


% HEADER AND FOOOTER 
%====================================
\newcommand\Header[1]{%
\begin{tikzpicture}[remember picture,overlay]
\fill[accentcol]
  (current page.north west) -- (current page.north east) --
  ([yshift=50pt]current page.north east|-current page text area.north east) --
  ([yshift=50pt,xshift=-3cm]current page.north|-current page text area.north) --
  ([yshift=10pt,xshift=-5cm]current page.north|-current page text area.north) --
  ([yshift=10pt]current page.north west|-current page text area.north west) -- cycle;
\node[font=\sffamily\bfseries\color{white},anchor=west,
  xshift=0.7cm,yshift=-0.32cm] at (current page.north west)
  {\fontsize{12}{12}\selectfont {#1}};
\end{tikzpicture}%
}

\newcommand\Footer[1]{%
\begin{tikzpicture}[remember picture,overlay]
\fill[lightcol]
  (current page.south east) -- (current page.south west) --
  ([yshift=-80pt]current page.south east|-current page text area.south east) --
  ([yshift=-80pt,xshift=-6cm]current page.south|-current page text area.south) --
  ([xshift=-2.5cm,yshift=-10pt]current page.south|-current page text area.south) --	
  ([yshift=-10pt]current page.south east|-current page text area.south east) -- cycle;
\node[yshift=0.32cm,xshift=9cm] at (current page.south) {\fontsize{10}{10}\selectfont \textbf{\thepage}};
\end{tikzpicture}%
}


%=====================================
%============================================================================%
%
%
%
%	DOCUMENT CONTENT
%
%
%
%============================================================================%
\newcommand{\beginleftcolumn}{\begin{leftcolumn}}
\newcommand{\myendleftcolumn}{\end{leftcolumn}}
\newcommand{\beginrightcolumn}{\begin{rightcolumn}}
\newcommand{\myendrightcolumn}{\end{rightcolumn}}
\newcommand{\badranratio}{0.29}
%default ratio was 0.31
\begin{document}

%\columnratio{0.31}
\setlength{\columnsep}{2.2em}
\setlength{\columnseprule}{4pt}
\colseprulecolor{white}


% LEBENSLAUF HIERE
%badranx removed footer
%\AtBeginShipoutFirst{\Header{CV}}%\Footer{1}}
%\AtBeginShipout{\AtBeginShipoutAddToBox{\Header{CV}}}%\Footer{2}}}
\AtBeginShipoutFirst{\Header{}}%\Footer{1}}
\AtBeginShipout{\AtBeginShipoutAddToBox{\Header{}}}%\Footer{2}}}
\newpage

\colseprulecolor{lightcol}
\columnratio{\badranratio}
\setlength{\columnsep}{2.2em}
\setlength{\columnseprule}{4pt}
\begin{paracol}{2}

%start badranx
%https://tex.stackexchange.com/questions/199382/how-to-relax-the-horizontal-spacing-rules-for-one-line/199383#199383
\begin{sloppypar}
%end badranx
%\begin{leftcolumn}
\beginleftcolumn
%---------------------------------------------------------------------------------------
%	META IMAGE
%----------------------------------------------------------------------------------------
%\includegraphics[width=\linewidth]{resources/image.jpg}	%trimming relative to image size

%---------------------------------------------------------------------------------------
%	META SKILLS
%----------------------------------------------------------------------------------------
	\fcolorbox{white}{white}{\begin{minipage}[c][1.5cm][c]{1\mpwidth}
		\LARGE{\textbf{\textcolor{maincol}{Yahya Badran}}} \\[2pt]
		\normalsize{ \textcolor{maincol} {Applied Mathematics for Network and Data Sciences (M.Sc.)} }
\end{minipage}} \\

\vspace{8pt}
%\cvsectionsmall{Contact}

\icontext{MapMarker}{16}{Mittweida, Germany}{black}\\%[6pt]
\iconemail{Envelope}{16}{Me@ybadran.com}{me@ybadran.com}{black}\\%[6pt]
\iconhref{Home}{16}{ybadran.com}{https://ybadran.com}{black}\\%[6pt]
\iconhref{Blogger}{16}{blog.ybadran.com}{https://blog.ybadran.com}{black}\\%[6pt]
\iconhref{Github}{16}{github.com/badranx}{https://www.github.com/badranx}{black}\\%[6pt]
\iconhref{StackOverflow}{16}{badran}{https://stackexchange.com/users/9942391/yijy?tab=accounts}{black}\\%[6pt]
\iconhref{Linkedin}{16}{linkedin.com/in/badranx}{https://linkedin.com/in/badranx}{black}\\%[6pt]
\icontext{Mobile}{16}{+4917676559158}{black}\\%[6pt]
%\iconhref{Xing}{16}{xing.com/user\_name}{https://www.xing.com/profile/User_Name}{black}\\
\vspace{-4pt}
%\cvsection{Skills}


\cvsectionsmallsmall{Skills: Programming}
\cvskillsection{Main:}
\cvtext{
 Python \textbullet{} C \textbullet{} C\#  \textbullet{} SQL \textbullet{} Bash
 }
% \cvsectionsmallsmall{1 year experience}
\cvskillsection{\\[3pt]Used in the past:}
\cvtext{
Java \textbullet{}  C++ \textbullet{} JavaScript/CSS/HTML
}
\cvskillsection{\\[3pt]Familiar:}
\cvtext{
Golang  \textbullet{} R \textbullet{} GLSL
}
 \cvsectionsmallsmall{Skills: Frameworks}
\cvtext{
Pytorch \textbullet{}  Numpy \textbullet{} Pandas \textbullet{} OpenGL \textbullet{} Flask \textbullet{} FastAPI \textbullet{} NLTK \textbullet{} Tensorflow  \textbullet{} scikit learn \textbullet{}  JAX (familiar) \textbullet{} Trax  \textbullet{} gitlab CI/CD  \textbullet{} Scrappy \textbullet{} React Native
\textbullet{} OpenCV }

\cvsectionsmallsmall{Skills: Tools}
\cvtext{
Linux \textbullet{} Git \textbullet{} Huggingface \textbullet{} AWS \textbullet{} MongoDB \textbullet{} Heroku     \textbullet{} Docker 
}
%\vspace{8pt}
\cvsectionsmallsmall{Languages}
\cvtext{
Arabic (C2)\textbullet{} English (C1) \textbullet{} German (B1) \textbullet{} Hebrew (A2)
}
%\cvskill{Arabic} {L1} {1} \\[-2pt]
%\cvskill{English} {C1} {0.9} \\[-2pt]
%%\cvsectionsmall{Language}
%\cvskill{German} {B1} {0.4} \\[-2pt]
%\cvskill{Hebrew} {A2} {0.3} \\[-2pt]
%%%References
\cvsectionsmallsmall{References} 
\cvtext{
\href{https://www.linkedin.com/in/mahbadran/}{\textbf{Mahmoud Badran}}, Senior Computer Vision Engineer, Reconess 
\\
\iconemail{Envelope}{16}{mbadran@reconess.com}{mbadran@reconess.com}{black}\\[6pt]
\href{https://www.linkedin.com/company/john-doe/}{\textbf{Tareq Al-Qab},  Senior Biomedical Engineer at Al-Najah University Hospital}
\\
\iconemail{Envelope}{16}{tareq.alqabb@gmail.com}{tareq.alqabb@gmail.com}{black}%\\[6pt]
}
\cvsectionsmallsmall{Soft skills}
%    Communication
%    Teamwork
%    Problem-solving
%    Time management
%    Critical thinking
%    Decision-making
%    Stress management
%    Adaptability
%    Conflict management
%    Creativity
%    Resourcefulness
%    Persuasion
%    Openness to criticism
\icontext{CaretRight}{12}{Critical thinking}{black}\\[2pt]
%\icontext{CaretRight}{12}{Teamwork}{black}\\[2pt]
\icontext{CaretRight}{12}{Openness to criticism}{black}%\\[3pt]
\icontext{CaretRight}{12}{Customer facing}{black}%\\[2pt]
%\icontext{CaretRight}{12}{Adaptability}{black}%\\[6pt]
%\newpage
%\cvsection{Education}
%\cvmetaevent
%{XX/XXXX - XX/XXXX}
%{Management Information Systems (M.Sc.)}
%{University of Lorem}
%{\textit{Field 1 • Field 2} \newline Master's thesis: \glqq title of Master's thesis\grqq.}
%\cvmetaevent
%{XX/XXXX - XX/XXXX}
%{Managment Information Systems (B.Sc.)}
%{University of Lorem}
%{\textit{Field 1 • Field 2} \newline Bachelor's thesis: \glqq title of Master's thesis\grqq.}
%
%\cvmetaevent
%{XX/XXXX - XX/XXXX}
%{A-Level}
%{High School Lorem}
%{\textit{Field 1 • Field 2 • Field 3 • Field 4 • Field 5}.}



%---------------------------------------------------------------------------------------
%	EDUCATION
%----------------------------------------------------------------------------------------
%
%\cvsection{Projekte}

%	\cvlist{
%		\item \hyperlink{https://github.com/philipempl/ether-twin}{\textbf{Ether-Twin.}}\\ Ethereum Applikation für Digital Twins.
%		\item \hyperlink{https://github.com/philipempl/Peter-Pan}{\textbf{Peter Pan.}}\\ Koch-App (t.b.a.).
%		\item \hyperlink{https://github.com/philipempl/Innovation-Tool}{\textbf{Innovation Tool.}}\\ Webcrawler für \hyperlink{https://ibi.de/}{Ibi}.
%		\item \hyperlink{https://github.com/philipempl/cozone}{\textbf{COZONE.}} \\ Soziales Netzwerk (t.b.a.).
%		\item \hyperlink{https://github.com/geritwagner/enlit}{\textbf{ENLIT.}}\\ Exploring new Literature (Bachelorarbeit).
%		\item \textbf{Crowdfunding.} \\Modul mit \hyperlink{https://senacor.com/}{Senacor} für \hyperlink{https://www.paydirekt.de/}{paydirekt}.
%		}

%\cvsection{Interests}
%
%\icontext{CaretRight}{12}{Guitarist}{black}\\[6pt]
%\icontext{CaretRight}{12}{DIY}{black}\\[6pt]
%\icontext{CaretRight}{12}{Fitness}{black}\\[6pt]
%\icontext{CaretRight}{12}{Cooking}{black}\\[6pt]
%\icontext{CaretRight}{12}{Travel}{black}\\[6pt]

%\cvqrcode{0.3}

%\end{leftcolumn}
\myendleftcolumn
%start badranx
\end{sloppypar}
%badranx end
\begin{sloppypar}
%\begin{\rightcolumn}
\beginrightcolumn
%---------------------------------------------------------------------------------------
%	TITLE  HEADER
%----------------------------------------------------------------------------------------


%---------------------------------------------------------------------------------------
%	PROFILE
%----------------------------------------------------------------------------------------
%\cvsection{Summary}
%\vspace{4pt}
%
%\cvtext{ I worked with many different technologies and programming languages. I did university studies in related to both software engineering and AI. Recently, I'm working on Machine learning problems with focus on NLP.
%
%My main interests are data science, computer graphics, and Linux. Both as a hobby and a profession.
%}

%---------------------------------------------------------------------------------------
%	WORK EXPERIENCE
%----------------------------------------------------------------------------------------

\vspace{10pt}
\cvsection{Education}
%\vspace{4pt}
\cvevent
{9/2018 - 7/2021}
	{M.Sc Applied Mathematics for Network and Data Sciences}
	{Mittweida University of Applied Science | Germany}
	{}
	{}
	\vfill\null

\cvsection{Work Experience}
%\vspace{4pt}
\cvevent
{8/2021 - today}
	{Data Science/Engineer (NLP)}
	{\href{www.langxlang.com}{LangXLang.com | non-profit | co-founder}}
	{Python, Docker, Stanza, Spacy, NLTK, Pytorch, Hugginface}
	{
	\begin{itemize}
    	\item Data pipelines and processing of multi-lingual corpora (text datasets).
    	\item Text generative models (mainly GPT2).
    	\item Seq2seq (translation) models.
    	\item Generating HTML/CSS and books from data.
	\end{itemize}
	}
	
	\vfill\null
\cvevent
{8/2021 - 6/2022}
	{Phd student of Machine learning}
	{Mittweida University of Applied Science | Germany}
	{}
	{}
	\vfill\null
	
\cvevent
{4/2016 – 9/2018}
	{Indie Game Developer}
	{\href{www.xlang.info}{xlang.info}}
	{C\#, Java, SQL, Mongodb, Python, Android}
	{
	\begin{itemize}
	\item Two libraries/APIs for video game developers.
	\item Customer support for the libraries.
	\item One Android project related to a unity library. Java.
	\item Two indie video Games. C\#
	\item One website for local users (Flask, SQL, MongoDB, HTML, CSS).
	\end{itemize}
	}
	\vfill\null

\cvevent
{6/2015 – 3/2016}
	{Software Developer}
	{PinchPoint | Ramallah, Palestine}
	{ C\#, Java, Javascript, REST API}
	{
	\begin{itemize}
	\item Worked on 3 different video games.
	\item Lead a team of interns to develop a video game.
	\item Added SIM card payment solution in the MENA region (REST API, Java EE, Javascript).
	\item Built two art creation tools/editors for our artists.
	\end{itemize}
	}
	\vfill\null


\cvevent
{1/2015 – 4/2015}
	{Software Developer}
	{Neiraba Animation Studios (intern) | West-bank, Palestine}
	{Javascript, HTML, CSS}
	{
	\begin{itemize}
	\item Converting Flash video games to Javascript/HTML5.
	\end{itemize}
	}
	\vfill\null

\cvsectionTight{Projects}
\begin{itemize}
\item  "Android \& Microcontrollers / Bluetooth" library: It was top 10 on the Assetstore input management category. Later, it became open-source and free.\\\techno{C\#, Java, Multi-threading, Bluetooth, Unity3d}
\begin{itemize}
\item Published on Unity: \href{https://assetstore.unity.com/packages/slug/16467}{u3d.as/78c}
\item Open-Source: \href{https://github.com/badranX/bt-lib}{github.com/badranX/bt-lib}
\end{itemize}
%\item  naive bayes nearest neighbor implementation. \footnotesize URL: \href{https://github.com/badranX/naive-bayes-nearest-neighbor}{github.com/badranX/naive-bayes-nearest-neighbor}
\vspace{2pt}
%\item MyGnuplot: Plotting library, interfacing GnuPlot. {\footnotesize URL: \href{https://github.com/badranX/MyPygnuplot}{github.com/badranX/MyPygnuplot}}
%\vspace{2pt}
%\item Opus-Raw: A Hugginface dataset. \footnotesize URL: \href{https://huggingface.co/datasets/badranx/opus_raw}{huggingface.co/datasets/badranx/opus\_raw}
%\vspace{2pt}
\item Transformer pytorch implementation. \footnotesize URL: \href{https://github.com/badranX/vanilla-transformer}{github.com/badranX/vanilla-transformer}
\vspace{2pt}
\item Generative models with vector quantization. deep learning. \footnotesize URL: \href{https://github.com/badranX/VQ-VAE-Expirements}{github.com/badranX/VQ-VAE-Expirements}
\end{itemize}
%\vfill\null
\vspace{6pt}
\cvsectionTight{Research}
\begin{itemize}[leftmargin=*]
\item Badran, Yahya. "VQ-VAE with Neural Gas and Fuzzy c-Means". Thesis. 2021.
\item \textbf{Acknowledged for demonstrating the last section in:} Demaine, Erik D., and Mikhail Rudoy. "Tree-Residue Vertex-Breaking: a new tool for proving hardness." 16th Scandinavian Symposium and Workshops on Algorithm Theory. 2018.
\item \textbf{Course research:} Badran, Yahya. 2022. “Non-linear Time Series Prediction: A Survey.” OSF Preprints. May 13. \href{https://doi.org/10.31219/osf.io/5ab8n}{doi:10.31219/osf.io/5ab8n}.
\item \textbf{Course research:} Badran, Yahya. 2022. “Variational Auto-encoder with Student's T Prior: Excerpt.” OSF Preprints. May 13. \href{https://doi.org/10.31219/osf.io/c2unf}{doi:10.31219/osf.io/c2unf}.
\end{itemize}
% hofixes to create fake-space to ensure the whole height is used
%\mbox{}
%\vfill
%\mbox{}
%\vfill
%\mbox{}
%\vfill
%\mbox{}
%\vfill
%\mbox{}
%\vfill
%\mbox{}
%\vfill
%\mbox{}


%Lorem, \today     \hspace{1cm}   \hrulefill

%\hspace*{30mm}\phantom{Lorem, \today }Philip Empl

%\end{\rightcolumn}
\myendrightcolumn
\end{sloppypar}
\end{paracol}
\end{document}
